\chapter{Another Example} 
\thispagestyle{fancy} %Resets custom page number location on first chapter page

\section{Big Files}
For big projects (like a thesis) it is sometimes helpful to separate your chapters indo different documents. This
can be done with the \verb=\input= command. Simply save your chapter as a separate .tex file without a header or
preamble, and then input it into the main dissertation file. Try to avoid using the \verb=\include= command as \LaTeX\
can get cranky with the formatting when ``including'' a .tex file.

\section{Non-Breaking Spaces}
The tilde character can be used to mark a non-breaking space. \LaTeX\ does a great job of lining words up, putting
in line breaks where appropriate, and making the text flow properly. However, there are some cases where you really
don't want \LaTeX\ to toss in a carriage return. The places that spring to mind are between numbers and their units,
and between Figure and Table references and their values. So for example, you don't want something like this in your
document:

\begin{quote}
	Blah blah blah blah blah blah blah blah blah is 10\\
	m from foo to bar.
\end{quote}

So always use the tilde when you use numbers with units, like 9.8~m~s$^{-2}$, and with references, like Figure~\ref{samplefig}.
This way \LaTeX\ will never put a line break between those elements.


\section{Hyperlinks}
You may have noticed a lot of blue text in this template. Those are internal hyperlinks that are automatically set up by 
the \texttt{hyperref} package. You can change the colors of the links by messing with the options in the preamble. For the
most part these links should set themselves up to work just fine by themselves. If you notice weird behavior you can usually
find solutions online. If not the use of the hyperlinks is purely for convenience when working with large documents and is not
required by the Grad College.

You can also use this package to include web links. Long links can sometimes escape the page margins though, so I recommend
giving them custom text: \href{https://grad.arizona.edu/gsas/dissertations-theses/dissertation-and-thesis-formatting-guides}
{A link to the UA Thesis guidelines.}

\section{Tables}
On the next page is a sample table, placed on the page by itself. Again, this table is pretty small, so it could 
probably just be placed on a page with text. To do this remove the \emph{p!} from the command. Working with tables in
\LaTeX\ is notoriously difficult, so make liberal use of StackExchange and Google. 

A comment on \texttt{deluxetable}. \texttt{deluxetable} is a special table package from the AAS formatting standards that
some people insist is much nicer than the default \LaTeX\ table. These people are wrong, and \texttt{deluxetable} is evil.
Therefore I have not included a way for this template to work with \texttt{deluxetable}. If you want to try, that's your
choice, but I cannot gurantee it won't break the template.

\begin{table}[p!]
\begin{center}
\caption[Short table caption for LOT]{Sample table caption (to appear with the actual table). \label{sampletable}}
\vspace{0.3in}
\begin{tabular}{ccc}
\hline 
\hline
Col A & Col B & Col C \\
\hline
1 & 2 & 3 \\
4 & 5 & 6\\
\hline
\end{tabular}
\end{center}
\end{table}

By the way, you can reference Tables like you can figures (Table~\ref{sampletable}).