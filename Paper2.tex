\chapter{Evidence for Gravity Waves in the Thermosphere of Saturn and Implications for Global Circulation} 
\thispagestyle{fancy} 

Originally published as: Brown, Zarah L. and Medvedev, Alexander S. and Starichenko, Ekaterina D. and Koskinen, Tommi T. and M\"{u}ller‐Wodarg, Ingo C. F. Evidence for Gravity Waves in the Thermosphere of Saturn and Implications for Global Circulation. \textit{Geophysical Research Letters} 49 (2022). doi: 10.1029/2021GL097219
[* Need to fix the references, add the figures, review the way the sections are set up, and double check the equations and other symbols/math.]


\section{Introduction}
\label{sec:intro}

Gravity waves (GWs) are ubiquitous in all convectively stable atmospheres. They are generated by various mechanisms (e.g., local convection, storms, instabilities and non-linearity of weather phenomena), which disturb the flow of air masses and propagate vertically, transporting energy and momentum. Upon propagation, the amplitudes of GWs grow exponentially in response to the decreasing density and become unstable at certain heights. There, GWs either break down or dissipate due to exponentially increasing molecular diffusion and thermal conduction, depositing their energy and momentum to the mean flow. Therefore, the main dynamical role of GWs is providing vertical coupling between the lower and upper atmospheric layers. GWs can greatly impact the thermal structure and dynamics of the middle and upper atmosphere.

GWs and their effects have been extensively studied observationally, theoretically and with modeling on Earth \cite<e.~g.>{Fritts_Alexander03, Yigit_Medvedev15} and in the atmospheres of other planets, mainly on Mars and Venus \cite<e.~g., see the recent review by>{MedvedevYigit19}. Much less is known about GWs in the atmospheres of outer planets. They have been detected on Jupiter with stellar occultations \cite{FrenchGierasch74, Yelle_etal96} and later directly measured by the Galileo probe \cite{Young_etal97}. On Saturn, GWs have been observed in the stratosphere \cite{Harrington_etal10} and lower ionosphere \cite{MatchevaBarrow12}. The Cassini Grand Finale observations of Saturn's thermosphere during ‘Deep Dip’ orbits in 2017 allowed the Ion and Neutral Mass Spectrometer (INMS; \citeA{Waite04}) to measure in-situ densities for the first time below 2,000 km. This revealed wave-like signatures with scales and amplitudes compatible with either GWs, or equatorial modes such as Kelvin and Rossby-gravity waves \cite{MW19}. Gravity waves have been proposed as a mechanism to redistribute energy from Saturn's auroral regions to lower latitudes, however, they must overcome the fast, high latitude, westward jets arising from Saturn’s ion drag and strong Coriolis force. If GWs can provide sufficient momentum to allow equatorward flow, they could help generate the higher than expected low latitude temperatures observed in the thermospheres and resolve the so-called ``energy crisis" \cite{MW19}. We address energy redistribution and its applicability to the other outer planets in the discussion.

The Cassini mission to Saturn from 2004 to 2017 observed many occultations by the upper atmosphere. Stellar occultations can provide some of the best vertical resolution relative to other remote sensing methods. Still, an attempt has not heretofore been made to fully characterize GWs in Cassini occultations, in part because some of the previously published occultation profiles have a vertical resolution below that which would allow for a robust characterization of GWs. Stellar occultations observed by the Ultraviolet Imaging Spectrograph \cite<UVIS,>{Esposito04} between January 14th, 2016 and August 4th, 2017 during the Grande Finale, however, have an overall improved vertical resolution (38 km on average) relative to previous UVIS observations. Importantly, these observations cover a wide latitude range (86$^\circ$S to 86$^\circ$N), which allows for an investigation of the global distribution of wave activity. Data processing and temperature retrievals are outlined in section~\ref{sec:retrievals} \cite<see>[for more detail]{Brown20}, and the results for GWs are described in section~\ref{sec:GWchar}. The results of the GW characterization have significant implications for circulation and redistribution of energy in the thermosphere, which we discuss in Sections 4 and 5.

\section{Density and Temperature Retrievals}
\label{sec:retrievals}

During a stellar occultation, UVIS measured the spectra of a UV-bright star as the intervening line of sight traversed regions of the atmosphere, ranging from completely unocculted to completely occulted. The UVIS extreme ultraviolet (EUV) channel covered wavelengths between 56 and 118 nm, but because interstellar hydrogen absorbs starlight at wavelengths shorter than about 91 nm these are not used in our analysis. Transmission tangent altitudes above the nearpoint are determined by the shortest distance between the line of sight and the 1 bar level based on the Cassini-fit gravitational potential \cite{A&S07} and radio occultation constraints. We derive transmission spectra as a function of tangent altitude between 91.1 and 118 nm by dividing the transmitted spectrum at each tangent altitude by the reference spectrum, which is the average of unocculted spectra observed at high altitudes above the atmosphere. The altitudes probed by the EUV occultations lie above the homopause, where each species behaves according to its own scale height and the abundances of species heavier than H\textsubscript{2} decline rapidly, leaving the thermosphere composed almost entirely of H\textsubscript{2}. The H\textsubscript{2} Lyman and Werner bands exhibit strong absorption features at the observed wavelengths.

The line of sight column density is a measure of the total amount of H\textsubscript{2} along the pathway through the thermosphere connecting the detector and the star. We determine line of sight column densities by fitting a forward model spectrum of H\textsubscript{2} to the observed transmission at each altitude using a Levenberg-Marquardt fitting algorithm. The high resolution forward model gives the expected transmission for a given H\textsubscript{2} abundance utilizing H\textsubscript{2} line lists \cite{Abgrall93a, Abgrall93b, Abgrall94, Abgrall00} and the line spread function of the UVIS instrument to convolve transmission. Because the H\textsubscript{2} cross section at these wavelengths depends on temperature, which we derive from the H\textsubscript{2} densities, we use an iterative process  \cite<see>[]{Koskinen21}.

We obtain number densities (in m\textsuperscript{-3}) by inverting line of sight column densities (in m\textsuperscript{-2}) using Tikhonov regularization, which reduces error at the expense of vertical resolution (i.e., being effectively a smoothing constraint). The average vertical resolution of these profiles (see Supporting Information Table T1) before the regularization is 14 km and 34 km for the final inversion. Such smoothing has the possibility of introducing wavelike features, and it is therefore important to verify that the waves that we detect are real and not an artifact of inversion. To make this determination, we examined the raw column density profiles and confirmed commensurate variations in density to those in the temperature profiles (see Supporting Information Figure S1). Since the column density profiles reflect real fluctuations in transmission as a function of altitude, this supports the detection of GWs in our observations.

The geometry of stellar occultations dictates the minimum horizontal length scale of observable waves. One can imagine the atmosphere as a series of locally isotropic concentric shells. A line of sight passes obliquely through successively deep layers of the atmosphere until it reaches the point with the shortest distance to the 1 bar surface level, the nearpoint, where the tangent altitude is calculated. After this point, the path passes through the ``back” of the atmosphere through the layers around successively higher altitudes. Because the H\textsubscript{2} density increases exponentially with depth, the greatest contribution to the line of sight density integration comes from the nearpoint. To estimate the effective length scale for these occultations, we calculated the contribution from each point along the line of sight to the column density integral. We find that, on average, 68.27\% of the signal comes from within 6,441 km of the nearpoint, with a standard deviation of 333 km. This indicates that the effective horizontal resolution of the data is about 6,400 km and waves with a shorter wavelength than this are not detectable.
\par

\section{Gravity Wave Characterization}
\label{sec:GWchar}
\subsection{Gravity Wave Profiles}
\label{sec:GW}

Gravity waves are disturbances of all field variables superimposed on the mean flow/state. Therefore, the retrieved vertical profiles $T(z)$ have to be separated into the mean temperature $\overline{T}$ and GW-induced perturbations $T^\prime=T-\overline{T}$. The bar denotes averaging over temporal and spatial scales much larger than wave phases. Since the profiles are almost instantaneous (with respect to wave periods), the partition can be performed only in vertical scales. 

There is no unique method of splitting the measured profiles into the mean and wave components. \citeA{JohnKumar13} and \citeA{Ehard_etal15} discussed several techniques for extracting GWs in the terrestrial atmosphere, while \citeA{Starichenko_etal21} explored them in the context of measurements on Mars. In this study, we apply the sliding-window least square polynomial fitting method of \citeA{Whiteway_Carswell95} modified as described in the paper of \citeA{Starichenko_etal21}. The profiles $\overline{T}(z)$ were obtained by fitting cubic polynomials within sliding 600-km windows with observational errors used for assigning a significance to the measurements at each point. The width of the interval was selected in order to resolve relatively small vertical-scale GW harmonics (shorter than a few density scale heights $H$), where $H$ varied with altitude from $\sim$50 to 150 km. The 600-km sliding intervals were shifted up from bottom to top by 110 km, and then the procedure was repeated from the top to bottom. The overlapping values of the polynomials were then averaged, and the resulting profiles smoothed by applying a moving average. Further details of the procedure can be found in the paper by \citeA<>[Section 3.2]{Starichenko_etal21}. Of the 22 Grand Finale occultations analyzed in \cite{Brown20}, three were not considered due to poor SNR at high altitudes, and one due to a high-altitude feature inconsistent with waves. Of the remaining 18 profiles, all had positive GW detections.

An example of the measured temperature profiles along with the fitted $\overline{T}(z)$ is shown in Figure~\ref{fig:Fig2}a. The gray shades indicate observational errors. The wave component $T^\prime(z)$ plotted in Figure~\ref{fig:Fig2}b represents a wave packet consisting of multiple harmonics rather than a single monochromatic wave. Since instantaneous distributions of phases cannot properly characterize the GW field, we determine the envelope, which defines the magnitude of wave fluctuations in the packet $|T^\prime| = \sqrt{\overline{T^{\prime 2}}}$ and is often called ``wave activity". It is calculated by performing Fourier decomposition in each sliding window and adding up contributions from all harmonics. Thus obtained amplitude is plotted in Figure~\ref{fig:Fig2}b with red dashed lines. If the waves propagated conservatively, the wave activity would grow exponentially with height following the exponential density decay. However, the rate of growth of the envelope decreases with altitude indicating that waves experience damping.
\begin{figure}
\noindent\includegraphics[width=\textwidth]{Fig2_v3.jpg}
\caption{Vertical profiles for the occultation measurement 2017-209-09-02-05. a) The measured (solid black) and fitted mean temperature (red dashed), gray shades denote observational uncertainties; b) wave temperature disturbance (solid black) and envelope (wave activity) (red dashed); c) Brunt-V\"ais\"al\"a frequency calculated for the mean (black) and measured temperature (red dashed); d) momentum flux (black, bottom axis, and produced momentum forcing (”wave drag”, red line, upper axis). The altitude is shown above the 1-bar pressure level.}
\label{fig:Fig2}
\end{figure}

In order to investigate the nature of wave damping, we examine the changes in  the squared Brunt-V\"ais\"al\"a frequency 
\begin{equation}
    N^2 = \frac{g}{T} \left(\frac{dT}{dz} + \frac{g}{c_p} \right)
    \label{eq:BVfreq}
\end{equation}
based on the measured ($T(z)$) and mean ($\overline{T}(z)$) temperature. Here $g$ is the acceleration of gravity and $c_p$ is the specific heat capacity at constant pressure. If the temperature gradient approaches or exceeds the superadiabatic lapse rate ($N^2$ approaches zero or becomes negative), convective instability takes place. While the atmosphere remains stable in general (black line), temperature disturbances may induce local instabilities. This is illustrated in Figure~\ref{fig:Fig2}c by the red line for $N^2(T,z)$, which drops below zero between $\sim$1300 and 1400 km. The convective instability limits the amplitude growth, induces wave saturation, and forces a deposition of wave momentum to the mean flow.

The vertical flux of horizontal momentum (per unit mass) carried by a GW harmonic can be calculated as $\mathbf{F} = (F_x, F_y, 0) = (\overline{u^\prime w^\prime}, \overline{v^\prime w^\prime}, 0)$, if the components of wave-induced fluctuations of velocity in the horizontal ($u^\prime, v^\prime)$ and vertical ($w^\prime)$ directions are known. They cannot be determined from a single temperature profile, however the estimate is possible for the absolute momentum flux $F= \sqrt{F_x^2 + F_y^2}$ \cite<e.g.,>[sect. 4]{Ern_etal04}: 
\begin{equation}
    F = \frac{1}{2} \frac{k_h}{m} \biggl(\frac{g}{N} \biggr)^2 
    \biggl(\frac{|T^\prime_{k,m}|}{\overline{T}} \biggr)^2. 
\label{eq:MF}
\end{equation}
The variables $k_h$ and $m$ in (\ref{eq:MF}) denote the horizontal and vertical wavenumbers,  $|T^\prime_{k,m}|$ is the amplitude of the corresponding harmonic, and the summation over all $m$ is implied. Vertical wavenumbers and amplitudes are known from the Fourier analysis, while the horizontal extent of the waves cannot be derived from the presented observations. Therefore, $k_h$ only serves as a scaling factor for momentum flux (\ref{eq:MF}) and momentum deposition (``wave drag") defined as
\begin{equation}
    a_h= -\frac{1}{\bar{\rho}} \frac{d\bar{\rho}F}{dz},
    \label{eq:drag}
\end{equation}
where $\bar{\rho}$ is the mean density. The subscript $h$ signifies that the momentum lost by waves provides horizontal acceleration/deceleration of the mean flow. However, the direction of this acceleration cannot be determined from a single profile. The main contribution to the line of sight column density in this observation (the densest atmospheric footprint at a target point) comes from the horizontal distance of around 6400 km (see Section 2). We used this number to scale the estimated momentum flux and wave drag by presetting the characteristic horizontal wavelength $\lambda_h^*=2\pi/k_h=6400$ km. Harmonics from the longer wavelength part of the spectrum may have contributed to the observed wave signature as well, however their input to the momentum flux decreases with $\lambda_h$. Smaller-scale harmonics, although unresolved by the instrument, likely also carry momentum. Therefore, the adopted $\lambda_h^*$ provides a reasonable estimate for the entire spectrum. The wave drag results underpinning both wind calculations depend on the assumed characteristic horizontal wavelengths.

The results of calculations for $F$ and $a_h$ are plotted in Figure~\ref{fig:Fig2}d. They help to elucidate the vertical structure of the GW, relate it to damping mechanisms and assess the momentum forcing imposed on the mean flow. The momentum flux $F$ (black line) does not grow with height monotonically, as would be expected in case of conservative propagation. It has at least three intervals where the amplitude ``saturates": between $\sim$900 and 950 km, 1050 and 1100 km, and 1300 and 1400 km. They are accompanied by three maxima of wave drag $a_h$ (red line), whose locations approximately coincide with the mentioned altitudes. A look at Figure~\ref{fig:Fig2}c shows that $N^2$ reaches local minima there, that is, the wave approaches the convective instability threshold. The magnitudes of the drag (up to a few thousand of m~s$^{-1}$~day$^{-1}$, where a Saturnian day is implied) are similar to those in individual measurements on Earth \cite{Yigit_Medvedev15} and Mars \cite{Starichenko_etal21}, although the day on Saturn is more than two times shorter. Overall, the analysis demonstrates that the behavior of the measured wave-like signatures is compatible with that of GWs.


\subsection{Spatial Distribution of Gravity Wave Activity}
\label{sec:spatial}

We next consider the spatial distribution of GW activity inferred from the measurements.

The measurements cover middle- to high latitudes (see Table S1 for occultation details).  Figure~\ref{fig:Maps} shows latitude-altitude cross-sections of the data in 10$^\circ$ latitude bins. Figure~\ref{fig:Maps}a presents the mean temperature $\overline{T}$. Across lines of constant altitude, temperatures peak near the auroral ovals (this is the case in pressure space as well, see Supporting Information Figure S2). The GW activity $|T'|$ (the envelope of wave packets) is plotted in Figure~\ref{fig:Maps}b. It shows that the activity generally grows with height and reaches $\sim$50~K in the upper portion of the domain. These magnitudes are much larger than in the thermospheres of Earth, Mars and Venus. However, the relative wave-induced perturbations of temperature $|T^\prime|/\overline{T}$ do not exceed $\sim$15\%, which is in line with measurements on these planets.

Note that in the upper thermosphere of Mars, the amplitudes of GW are even larger and can reach up to 40\% \cite{Yigit_etal21}. The GW activity tends to be greater in the southern hemisphere and we speculate that this is a seasonal effect. The observations were made just after the northern summer solstice with permanent night poleward of about 60$^\circ$S.

Another useful characteristic of the GW activity is the wave potential energy (per unit mass)
\begin{equation} 
E_p = \frac{1}{2} \left(\frac{g}{N}\right)^2 \left(\frac{|T^\prime|}{\overline{T}} \right)^2, 
\label{eq:GW-Ep}
\end{equation}
which is shown in Figure~\ref{fig:Maps}c. Being a quadratic quantity of $|T^\prime|/\overline{T}$, $E_p$ also grows with height, however the magnitudes are about two orders of magnitude larger than those on Mars \cite<>[Figure 7b]{Starichenko_etal21}. A closer look shows that the difference is due to the coefficient $(g/N)^2$, where $g$ is larger and $N^2$ is, on average, two times smaller on Saturn. The latter indicates that the Saturnian thermosphere is less convectively stable than the atmospheres of Earth, Mars and Venus (above the cloud top).


%\par
\begin{figure}
%\noindent\includegraphics[width=\textwidth]{Fig3_v1.png}
\noindent\includegraphics[width=\textwidth]{Maps_1.png}
%\centerline{\noindent\includegraphics[width=16.0cm]{Maps_1.png}}
\caption{Latitude-altitude cross-sections of the a) mean temperature $\overline{T}$, b) GW amplitudes $|T^\prime|$ (in K), c) wave potential energy (per unit mass) $E_p$ (in J~kg$^{-1})$, d) vertical flux of horizontal wave momentum (per unit mass), or ``wave momentum" (in m$^2$~s$^{-2}$), e) acceleration of the mean flow (``GW drag", in m~s$^{-1}$~day$^{-1}$), and f) mean meridional velocity $\bar{v}^*$ (in m~s$^{-1}$). The size of the latitudinal bins is 10$^\circ$.}
\label{fig:Maps}
\end{figure}

\subsection{Derived Gravity Wave Characteristics and Meridional Transport}
\label{sec:GWD}

Having considered the retrieved quantities, we now turn our attention to the derived ones. They include wave momentum flux $F$ and GW drag $a_h$. Calculations of $F$ produced spurious solutions in a few vertical profiles in the lower part of the domain, where amplitudes are small. They occurred due to numerical errors (large differences of small values), which, in all cases, lead to nonphysically steep growth of $F$ with height. To correct for this drawback, we applied an adjustment procedure, as described in Supporting Information Text S1. The results are shown in Figures~\ref{fig:Maps}d,e,f. Wave momentum fluxes cease their exponential growth at all latitudes and even locally decay with height in some observations. This points to a vertical damping of GWs, which results in momentum forcing of the ambient flow. Figure~\ref{fig:Maps}e gives the first insight into the distribution of the GW drag in the Saturnian thermosphere. It has a distinctive latitudinal structure with maximum accelerations exceeding 500 m~s$^{-1}$~day$^{-1}$ in middle and high latitudes. In the vertical, the regions of maximum drag are located between 1300 and 1500 km. This distribution provides evidence of GW breaking/saturation processes in a relatively narrow altitude range, similar to what occurs near the mesopause on Earth and Mars.

The precise horizontal direction of wave propagation, and therefore of GW drag, cannot be inferred from a single temperature profile. Assuming that packets propagate in all directions, one can anticipate that the obtained drag characterizes the forcing in the zonal direction as well, and assign it to $a_x$. This allows for estimating the mean meridional transport (residual) velocity $\bar{v}^*$, as defined in the Transformed Eulerian Mean (TEM) formulation \cite{Andrews_etal87}. The residual velocity describes transport of tracers, including temperature. For fast rotating planets, the scaling of the zonal momentum equation gives an approximate balance between the Coriolis force and GW drag $2\Omega\sin\phi\bar{v}^* \approx - a_x$ \cite<>[Sect. 3.2]{MedvedevHartogh07}, where $\phi$ is the latitude, and $\Omega$ is the rotational frequency of the planet. This balance holds away from the equatorial region, where $\sin\phi$ tends to zero. The calculated magnitudes of $\bar{v}^*$ are shown in Figure~\ref{fig:Maps}f. The distribution, generally, coincides with that of the GW drag. At altitudes between 1300 and 1500 km, $\bar{v}^*$ exceeds 60 m~s$^{-1}$, which is about ten times larger than on Earth or Mars. Given the Saturn radius and assuming that the velocity does not change sign, this implies that a parcel of air would take $\sim$37 Earth days to move between the pole and equator. Thus, our measurements point out to a rapid horizontal exchange of mass and temperature in the thermosphere of Saturn.

\section{Modified Geostrophy and the Impact of Gravity Wave Drag}
\label{sec:modgeo}

The zonal wind $\bar{u}$ itself cannot be estimated from GW properties and temperature alone using the gradient wind relation, because its boundary conditions are not known. However, since the Grand Finale observations provide both temperature and density as a function of height, $\bar{u}$ can be calculated as a function of the gravitational potential along surfaces of constant pressure \cite{Brown20} using modified geostrophy \cite{LW95}. This approach assumes a balance between the Coriolis force, meridional pressure gradient and auroral ion drag, self-consistently determining both zonal and meridional velocities. If the ion drag is neglected, the diagnostics converts into two independent approaches: the gradient wind balance for the zonal component and the TEM technique for the meridional transport discussed above. The application of modified geostrophy to the UVIS observations has been described in detail by \citeA<>[see the Methods section]{Brown20}, where the diagnosed winds at three pressure levels were first presented. 

Here we extend this approach to include the GW drag and present the results on the entire altitude-latitude plane (See Supporting Information Text S2). Temperatures and pressures were used to calculate geostrophic zonal winds $u_g$ \cite<>[eqn. 2]{Brown20}. The modified geostrophic zonal wind $u$ \cite<>[eqn. 7]{Brown20} depends mainly on meridional gradients of temperature and density. The winds estimated without GWs, shown in Figure~\ref{fig:MG}a, are fast and retrograde (westward) in both hemispheres, reaching hundreds of m~s$^{-1}$. Unlike the impact on the meridional wind, adding $a_x$ with either sign, alters the zonal wind by only a few m~s$^{-1}$ (Figure~\ref{fig:MG}b). Because the westward zonal jets are more than an order of magnitude faster than peak meridional wind speeds, the GW drag is most likely directed against the mean wind, i.e., acts in an eastward direction and has a positive sign.

\begin{figure}
\noindent\includegraphics[width=\textwidth]{LnW_Wind_Contour_Quad_revised.png}
\caption{Latitude-altitude cross-sections calculated under the modified geostrophy approximation: a) zonal wind without and b) including the GW forcing, c) meridional wind without GW drag and d) with GW drag included.}
\label{fig:MG}
\end{figure}

Without GW forcing, meridional winds exhibit poleward flow above $\sim$1000 km in each hemisphere at latitudes poleward of approximately $\pm70^\circ$ and much weaker equatorward transport at latitudes below (Figure~\ref{fig:MG}c). Such circulation is maintained by ion friction, whose peak coincides with the maximum of Joule heating in the auroral region. The addition of GW drag produces noticeable changes in the distribution and strength of the meridional winds (Figure~\ref{fig:MG}d). Peak equatorward winds increase from 34 to 57 m~s$^{-1}$ in the Northern hemisphere and from 34 to 87 m~s$^{-1}$ in the South, with a shift to lower latitudes and higher altitudes in the southern hemisphere. Furthermore, the equatorward winds do not decay steeply away from the auroral region as in Figure~\ref{fig:MG}c, but extend farther to lower latitudes. The overall effect of eastward GW drag is greater transport from the auroral latitudes towards the equator. We note that the meridional winds estimated here under the modified geostrophy approximation are compatible with those derived with the TEM method (Figure~\ref{fig:Maps}f). The differences between them in the polar regions are due to contribution of ion drag, which is not accounted for in the TEM approach.

\section{Discussion}
\label{sec:diss}
The GW-induced enhancement of the meridional circulation found in our analysis is important for understanding energy transport in the thermosphere and in resolving the so-called outer planet ``energy crisis". First introduced in the 1970s, this term describes the observation that the upper atmospheres of all Solar system outer planets are hotter than expected based on radiative heating alone by a few hundred degrees. Enough energy is deposited at auroral latitudes by Joule heating to explain those temperatures at all latitudes, however fast moving westward jets arising from Saturn’s strong Coriolis force and auroral ion drag act to inhibit equatorward redistribution.

In search for a mechanism to facilitate the transport of heat toward the equator and using general circulation modeling, \citeA{MW19} implemented a parameterized wave drag based on characteristics of low-latitude waves observed by INMS to the Saturn Thermosphere Ionosphere Model (STIM) \cite{MW12}. They showed that this drag was sufficient to slow down the fast zonal jets and allow heat to be transported to low latitudes, increasing temperatures there to observed values. The GW forcing that we derive here from observations is of the same order of magnitude as those applied to STIM  \cite<see Figure 3 of>{Brown20}.

Can this GW-mediated transport occur on the other outer planets? While the magnetic fields of Uranus and Neptune differ substantially from Saturn's, Jupiter has polar auroral ovals and the circulation in its thermosphere is expected to be similar to Saturn \cite{SA09}. Observations of H$_3^+$ emissions were recently used to construct global maps of the Jovian thermosphere, which exhibit decreasing temperature from the poles to the equator in the quiescent state \cite{O'Donoghue21} similar to the meridional temperature gradient reported for Saturn \cite{Brown20}. Based on this, the authors argued that Coriolis forces (and other effects that constrain auroral energy to the poles) are overcome at Jupiter, but did not identify a mechanism that would facilitate that. By deriving accelerations and winds from observed gravity waves that are comparable to model predictions, our study demonstrates that the momentum deposited by the observed GWs can provide the dynamical forcing sufficient for maintaining an intense equatorward heat transport in the Saturnian thermosphere. Given that GWs have also been detected on Jupiter \cite{Young_etal97}, they could play a similar role in the Jovian thermosphere.

\section{Conclusions}
\label{sec:concl}
%Text here ===>>>
We used the stellar occultation data from the Cassini Ultraviolet Imaging Spectrograph (UVIS) obtained during the Grand Finale to derive vertical profiles of temperature disturbances in the thermosphere of Saturn. Among them, 18 profiles were fully consistent with the physics of gravity waves (GWs) and, therefore, represent GW signatures in temperature and density. The observations covered middle and high latitudes of both hemispheres, which allowed for exploring the spatial distribution of GW activity for the first time. The following main conclusions have emerged from this study.
\begin{enumerate}
    \item The observations provide evidence for the ubiquitous presence of GW packets in the thermosphere of Saturn between $\sim$600 and 1600 km above the 1 bar level.
    \item The amplitudes of GW-induced temperature fluctuations are on the order of tens of Kelvin and can exceed 50 K in the observations.
    \item The associated wave potential energy (per unit mass) is up to $5\cdot 10^{4}$ J~kg$^{-1}$, and the vertical flux of horizontal momentum (per unit mass) is up to 2000 J~kg$^{-1}$. For comparison, these quantities are about 100 times larger than for GWs on Earth and Mars.
    \item The momentum deposited by breaking/dissipating GWs (``wave drag") to the ambient flow provides acceleration/deceleration of the order of hundreds of m~s$^{-1}$~day$^{-1}$.
    \item Such GW drag enhances the meridional transport in both hemispheres approaching 100 m~s$^{-1}$ directed from the high-latitude auroral regions towards the equator. Drag of a similar magnitude was found in simulations to be sufficient for driving the equatorward meridional transport and explaining the observed thermospheric temperatures \cite{MW19}.
\end{enumerate}

The presented findings support auroral heating with equatorward redistribution as an important mechanism to explain the observed temperatures in Saturn's thermosphere. This mechanism can be applicable at Jupiter and more broadly to giant planets with magnetic field configurations similar to these gas giants.



\acknowledgments
Z.L.B and T.T.K. acknowledge support by the NASA Cassini Data Analysis Program Grant (80NSSC19K0902). E.D.S. acknowledges support by the Russian Ministry of Science and Higher Education. I.M-W. acknowledges support by the UK Science and Technology Facilities Council (STFC) grant (ST/W001071/1).
